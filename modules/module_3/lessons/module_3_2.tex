\PassOptionsToPackage{unicode=true}{hyperref} % options for packages loaded elsewhere
\PassOptionsToPackage{hyphens}{url}
%
\documentclass[]{article}
\usepackage{lmodern}
\usepackage{amssymb,amsmath}
\usepackage{ifxetex,ifluatex}
\usepackage{fixltx2e} % provides \textsubscript
\ifnum 0\ifxetex 1\fi\ifluatex 1\fi=0 % if pdftex
  \usepackage[T1]{fontenc}
  \usepackage[utf8]{inputenc}
  \usepackage{textcomp} % provides euro and other symbols
\else % if luatex or xelatex
  \usepackage{unicode-math}
  \defaultfontfeatures{Ligatures=TeX,Scale=MatchLowercase}
\fi
% use upquote if available, for straight quotes in verbatim environments
\IfFileExists{upquote.sty}{\usepackage{upquote}}{}
% use microtype if available
\IfFileExists{microtype.sty}{%
\usepackage[]{microtype}
\UseMicrotypeSet[protrusion]{basicmath} % disable protrusion for tt fonts
}{}
\IfFileExists{parskip.sty}{%
\usepackage{parskip}
}{% else
\setlength{\parindent}{0pt}
\setlength{\parskip}{6pt plus 2pt minus 1pt}
}
\usepackage{hyperref}
\hypersetup{
            pdftitle={module\_3\_2 - Visualization Exercise},
            pdfauthor={Keaton Wilson},
            pdfborder={0 0 0},
            breaklinks=true}
\urlstyle{same}  % don't use monospace font for urls
\usepackage[margin=1in]{geometry}
\usepackage{graphicx,grffile}
\makeatletter
\def\maxwidth{\ifdim\Gin@nat@width>\linewidth\linewidth\else\Gin@nat@width\fi}
\def\maxheight{\ifdim\Gin@nat@height>\textheight\textheight\else\Gin@nat@height\fi}
\makeatother
% Scale images if necessary, so that they will not overflow the page
% margins by default, and it is still possible to overwrite the defaults
% using explicit options in \includegraphics[width, height, ...]{}
\setkeys{Gin}{width=\maxwidth,height=\maxheight,keepaspectratio}
\setlength{\emergencystretch}{3em}  % prevent overfull lines
\providecommand{\tightlist}{%
  \setlength{\itemsep}{0pt}\setlength{\parskip}{0pt}}
\setcounter{secnumdepth}{0}
% Redefines (sub)paragraphs to behave more like sections
\ifx\paragraph\undefined\else
\let\oldparagraph\paragraph
\renewcommand{\paragraph}[1]{\oldparagraph{#1}\mbox{}}
\fi
\ifx\subparagraph\undefined\else
\let\oldsubparagraph\subparagraph
\renewcommand{\subparagraph}[1]{\oldsubparagraph{#1}\mbox{}}
\fi

% set default figure placement to htbp
\makeatletter
\def\fps@figure{htbp}
\makeatother


\title{module\_3\_2 - Visualization Exercise}
\author{Keaton Wilson}
\date{3/11/2020}

\begin{document}
\maketitle

\begin{center}\rule{0.5\linewidth}{\linethickness}\end{center}

\hypertarget{part-1---a-visualization-primer---45-minutes}{%
\subsection{Part 1 - A Visualization Primer - (45
minutes)}\label{part-1---a-visualization-primer---45-minutes}}

Some inspiring figures:\\
\url{http://www.r-graph-gallery.com/19-map-leafletr/}~\\
\url{http://www.r-graph-gallery.com/21-distribution-plot-using-ggplot2}~\\
\url{http://www.r-graph-gallery.com/274-map-a-variable-to-ggplot2-scatterplot/}~\\
\url{http://www.r-graph-gallery.com/123-circular-plot-circlize-package-2/}

\hypertarget{a.-why-do-visualizations-matter---small-group-activity}{%
\subsubsection{1a. Why do visualizations matter? - Small Group
Activity}\label{a.-why-do-visualizations-matter---small-group-activity}}

\begin{itemize}
\tightlist
\item
  Take 3 minutes individuals - write down your top 10 reasons why data
  visualization is important or useful to you?
\item
  Take 5 minutes with your group - compare your notes, come to a group
  consensus - rank the top 5 reasons from the group.
\item
  Have your \emph{Presenter} write the group's reasons on the board, and
  we'll spend some time discussing.
\end{itemize}

\hypertarget{b.-types-of-visualization---big-thinking-about-data}{%
\subsubsection{1b. Types of visualization - big thinking about
data}\label{b.-types-of-visualization---big-thinking-about-data}}

\hypertarget{b.1-small-group-activity}{%
\paragraph{1b.1 Small group activity}\label{b.1-small-group-activity}}

Within your group - what are all of the types of visualization you can
think of. The Conceptor will draw them initially, then I'll draw them on
the board during a speak out.

\hypertarget{b.2-activity---matching-data-types-to-visualization-types}{%
\paragraph{1b.2 Activity - matching data types to visualization
types}\label{b.2-activity---matching-data-types-to-visualization-types}}

Matching activity in group - can you match the visualization to the data
type.

File for activity is in the images folder.

\hypertarget{c-figure-critique---what-makes-a-good-data-visualization}{%
\subsubsection{1c Figure Critique - What makes a good data
visualization?}\label{c-figure-critique---what-makes-a-good-data-visualization}}

\begin{itemize}
\tightlist
\item
  Small mini-lecture on principles of data visualization and design.

  \begin{enumerate}
  \def\labelenumi{\arabic{enumi}.}
  \tightlist
  \item
    Some resources to build this:
  \end{enumerate}

  \begin{enumerate}
  \def\labelenumi{\alph{enumi}.}
  \tightlist
  \item
    \url{https://www.fusioncharts.com/whitepapers/downloads/Principles-of-Data-Visualization.pdf}
  \item
    \url{https://moz.com/blog/data-visualization-principles-lessons-from-tufte}
  \item
    \url{http://stat.pugetsound.edu/courses/class13/dataVisualization.pdf}
  \item
    \url{http://paldhous.github.io/ucb/2016/dataviz/week2.html}
  \end{enumerate}
\end{itemize}

Figures:

\begin{enumerate}
\def\labelenumi{\arabic{enumi}.}
\item
  \includegraphics{https://raw.githubusercontent.com/keatonwilson/biostats/master/Images/3d-column-chart-of-awesome.png}
\item
  \includegraphics{https://raw.githubusercontent.com/keatonwilson/biostats/master/Images/Bling_fig3.jpg}
\item
  \includegraphics{https://raw.githubusercontent.com/keatonwilson/biostats/master/Images/cawley_fig1.jpg}
\item
  \includegraphics{https://raw.githubusercontent.com/keatonwilson/biostats/master/Images/roeder_fig4.jpg}
\end{enumerate}

\begin{itemize}
\tightlist
\item
  Students will examine and critique figures from primary literature in
  their groups. Each group will be passed a figure, students will take
  10 minutes to try to interpret the figure, and generate a list of
  problems. Each group will also draw and present a reimagined version
  of the figure that is better.
\end{itemize}

\end{document}
